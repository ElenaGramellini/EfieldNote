%%%%%%%%%%%%%%%%%%%%%%%%%%%%%%%%%%%%%%%%%%%%%%%%%%%
\section{Results}\label{sec:Results}
%%%%%%%%%%%%%%%%%%%%%%%%%%%%%%%%%%%%%%%%%%%%%%%%%%%
The following table summarizes the results from the two methods described above.
\begin{center}
\begin{table}[htb]
  \begin{center}
    \begin{tabular}{c|c|c|c}
      \multicolumn{4}{c}{\textbf{Summary of Results}} \\
      \hline \hline
       Method & Drift Time [$\mu$s] & Drift velocity [mm/$\mu$s]  & Electric field [V/cm]\\
       \hline
       Electric circuit        & 311.3  & 1.51 & 486.5 \\
       \hline
       Crossing Tracks: Run I  & 311.0 $\pm$ 2.5 & 1.51 $\pm$0.01 & 487 $\pm$ 21 \\
       \hline
       Crossing Tracks: Run II & 315.6 $\pm$ 2.7 & 1.49 $\pm$ 0.01 & 468.3 $\pm$ 21  \\
       \hline
       \hline
       \end{tabular}%}
    \caption{Summary of the results. }
    \label{tab:ResultsFinal}
    \end{center}
\end{table}
\end{center}

We find for the 90.3 K temperature of the argon in the center of the cryostat an electric field value of 486.5 V/cm using the single line drawing with a drift velocity of 1.51 mm/$\mu s$ and a drift time of 311.3 $\mu s$. Results consistent with these values are found as well using cathode to anode piercing tracks although with a larger uncertainty in the calculation.
